%%%
% Plantilla de Trabajo
% Modificación de una plantilla de Latex de Frits Wenneker para adaptarla 
% al castellano y a las necesidades de escribir informática y matemáticas.
%
% Editada por: Mario Román
%
% License:
% CC BY-NC-SA 3.0 (http://creativecommons.org/licenses/by-nc-sa/3.0/)
%%%

%%%%%%%%%%%%%%%%%%%%%%%%%%%%%%%%%%%%%%%%
% Short Sectioned Assignment
% LaTeX Template
% Version 1.0 (5/5/12)
%
% This template has been downloaded from:
% http://www.LaTeXTemplates.com
%
% Original author:
% Frits Wenneker (http://www.howtotex.com)
%
% License:
% CC BY-NC-SA 3.0 (http://creativecommons.org/licenses/by-nc-sa/3.0/)
%
%%%%%%%%%%%%%%%%%%%%%%%%%%%%%%%%%%%%%%%%%

%----------------------------------------------------------------------------------------
%	PAQUETES Y CONFIGURACIÓN DEL DOCUMENTO
%----------------------------------------------------------------------------------------

%%% Configuración del papel.
% fourier: Usa la fuente Adobe Utopia. (Comentando la línea usa la fuente normal)
\documentclass[paper=a4, fontsize=11pt, spanish]{scrartcl} 
\usepackage{fourier}

% Centra y formatea los títulos de sección.
% Quita la indentación de párrafos.
\usepackage{sectsty} % Allows customizing section commands
\allsectionsfont{\centering \normalfont\scshape} % Make all sections centered, the default font and small caps
\setlength\parindent{0pt} % Removes all indentation from paragraphs - comment this line for an assignment with lots of text

% Permite elegir cabeceras y pies de página.
\usepackage{fancyhdr} % Custom headers and footers
\pagestyle{fancyplain} % Makes all pages in the document conform to the custom headers and footers
\fancyhead{} % No page header - if you want one, create it in the same way as the footers below
\fancyfoot[L]{} % Empty left footer
\fancyfoot[C]{} % Empty center footer
\fancyfoot[R]{\thepage} % Page numbering for right footer
\renewcommand{\headrulewidth}{0pt} % Remove header underlines
\renewcommand{\footrulewidth}{0pt} % Remove footer underlines
\setlength{\headheight}{13.6pt} % Customize the height of the header


%%% Castellano.
% noquoting: Permite uso de comillas no españolas.
% lcroman: Permite la enumeración con numerales romanos en minúscula.
% fontenc: Usa la fuente completa para que pueda copiarse correctamente del pdf.
\usepackage[spanish,es-noquoting,es-lcroman]{babel}
\usepackage[utf8]{inputenc}
\usepackage[T1]{fontenc}
\selectlanguage{spanish}


%%% Matemáticas.
% Paquetes de la AMS. Para entornos de ecuaciones.
\usepackage{amsmath,amsfonts,amsthm}

% Incluye números entre secciones y ecuaciones.
\numberwithin{equation}{section} % Number equations within sections (i.e. 1.1, 1.2, 2.1, 2.2 instead of 1, 2, 3, 4)
\numberwithin{figure}{section} % Number figures within sections (i.e. 1.1, 1.2, 2.1, 2.2 instead of 1, 2, 3, 4)
\numberwithin{table}{section} % Number tables within sections (i.e. 1.1, 1.2, 2.1, 2.2 instead of 1, 2, 3, 4)

% Diagramas conmutativos
\usepackage{tikz}
\usetikzlibrary{matrix,arrows}
\tikzset{every loop/.style={min distance=10mm,in=0,out=60,looseness=10}}
\usepackage{tikz-cd}
\usetikzlibrary{cd}


%----------------------------------------------------------------------------------------
%	TÍTULO
%----------------------------------------------------------------------------------------
% Título con las líneas horizontales, nombres y fecha.

\newcommand{\horrule}[1]{\rule{\linewidth}{#1}} % Create horizontal rule command with 1 argument of height

\title{
  \normalfont \normalsize 
  \textsc{Universidad de Granada.} \\ [25pt] % Your university, school and/or department name(s)
  \horrule{0.5pt} \\[0.4cm] % Thin top horizontal rule
  \huge Qubits \\ % The assignment title
  \horrule{2pt} \\[0.5cm] % Thick bottom horizontal rule
}

\author{Ignacio Cordón y Mario Román} % Your name

\date{\normalsize\today} % Today's date or a custom date



%----------------------------------------------------------------------------------------
%	DOCUMENTO
%----------------------------------------------------------------------------------------


\begin{document}
\maketitle % Escribe el título

\section {El producto tensorial}
  \subsection {Funciones multilineales}
    En un espacio vectorial, una \textbf{función lineal} es aquella que respeta sumas y productos por escalares.
    Es decir, respeta la estructura de espacio vectorial:
    $$ f(v+w) = f(v) + f(w) $$
    $$ f(\lambda v) = \lambda f(v) $$

    Al intentar generalizar este concepto a una función en varias variables, podríamos llegar a las funciones 
    lineales sobre el espacio producto. Por ejemplo, una $f$ lineal sobre el espacio $V \times W$ cumple:
    $$ f(v+v',w+w') = f(v,w) + f(v',w')$$
    $$ f(\lambda v, \lambda w) = \lambda f(v,w) $$
    Pero esta estructura no es "lineal sobre cada variable". Funciones como el producto tienen otra
    estructura:
    $$ (A+B)(C+D) = AC+AD+BC+BD \neq AC+BD $$
    
    La otra forma de generalizar las funciones lineales son las funciones \textbf{multilineales}, que serán lineales
    en cada una de las variables por separado y cuando lo demás queda fijo. Cumplen:
    $$ f(v+v',w) = f(v,w) + f(v',w) $$
    $$ f(v,w+w') = f(v,w) + f(v,w') $$
    $$ f(\lambda v,w) = \lambda f(v,w) = f(v,\lambda w) $$

  \subsection{Estructura del espacio producto tensorial}
    El espacio producto tensor de dos espacios vectoriales se notará por $V \otimes W$. Sus vectores serán de
    la forma $v \otimes w$, donde $v \in V$, $w \in W$ y donde $\otimes$ se comportará como una función multilineal.
    Es decir:
    $$ v \otimes w + v' \otimes w  = (v+v') \otimes w $$
    $$ v \otimes w + v \otimes w'  = v \otimes (w+w') $$
    $$ (\lambda v) \otimes w = w \otimes (\lambda v)  $$
    El espacio $V \otimes W$ tendrá como base todos los posibles productos tensoriales entre un vector de $V$ y otro
    de $W$. Esto hará que su dimensión sea el producto de las dimensiones:
    $$ dim(V \otimes W) = dim(V)dim(W) $$
    
    Por ejemplo, $\mathbb{R}^2\otimes\mathbb{R}^2$ tiene dimensión cuatro y base:
    $$ e_1\otimes e_1,\quad e_1\otimes e_2,\quad  e_2 \otimes e_1,\quad  e_2 \otimes e_2 $$
    Nótese aquí que no todos los elementos del espacio van a poder escribirse como un producto tensorial
    de vectores de los factores, por ejemplo $e_1 \otimes e_1 + e_2 \otimes e_2$ no puede escribirse de forma
    más reducida. %% ¿¿POR QUÉ?? -> a^2 + b^2 no puede expresarse como producto de sumas de a y b
    
  \subsection{Propiedad universal del producto tensorial}
    El producto tensorial cumple el siguiente diagrama conmutativo.
    \begin{center}
       \begin{tikzcd}
         {V \otimes W} \arrow{r}{\exists! \tilde f} & Z \\
	 {V \times W} \arrow{u}{\otimes} \arrow{ur}[swap]{f}
       \end{tikzcd}
      \end{center}
    En el que para cualquier función multilineal $f$, existe una única función
    lineal $\tilde f$ que lo hace conmutativo. Es decir, toda función multilineal es la proyección 
    al espacio producto tensorial seguida de una función lineal.
    
\end{document}